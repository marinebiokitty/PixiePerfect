\documentclass[char]{PP}
\parindent=0pt
\begin{document}
\name{\cFWanabe{}}

You are \cFWanabe{} (\cFWanabe{\They}/\cFWanabe{\Them}), and an unhappy member of the Court of Flora and Fauna. Your specialty is in Agarics (a subset of mushrooms, often with red caps), and you feel every bit as overlooked as they are. Fungi aren’t even in the name of your court!

At first, the Court was great. You remember fondly the first few years when you were welcomed, and all of the Pixies of your court showed great interest in your special talent. But that novelty faded, and you were left very much alone and ignored. But no more! You have a two-part plan to earn the attention and acclaim you deserve in the short term, and ultimately a more permanent solution.

Part one involves the Tree. The root of all Pixie Magic is Pixie Dust, which comes from the Tree of Time, and is managed by the Magic Pixes of the Court of Makers and Magic. If something were to happen to the tree, everyone, and everything would be thrown into crisis, and whoever fixed it would be the hero of all of Pixie Hollow for decades to come. And it just so happens that Agarics are well suited to both poisons and cures. So you made one of each

Ultimately, you had to sneakily “borrow” some of the season elements (5 units to be exact), but the Element pixies always make plenty extra, so it was no big deal. You also needed to “borrow” a little extra pixie dust (4 units to be exact), but again, the Magic pixies always keep a bunch spare in case of an emergency. They’ll never miss the few units you borrowed.

With both the poison and the antidote prepared, you snuck down to the deep roots of Pixie Hollow, and administered the poison, pouring it out among the roots. Then you went about your life, the antidote in your pocket, ready for use the moment someone raised the alarm. It’s been \textbf{over three months} since you enacted your plan, but finally, just this afternoon, \cMTree{} went rushing by you, mumbling something under their breath about dust account books. It was time! And on the day of the portal opening no less. How perfect!

You reached into your pocket, your moment so close you could taste it. Only to find that your pocket was empty, and your fingers were sticking through a little hole at the bottom! The antidote (clear bottle, dark blue liquid) was gone! You have no idea where you could have lost it, or when. Of course you still have the recipe, you could make another antidote, but it would require new supplies. 

Well, uh, while you figure out how to rescue part one, you might as well move forward with part two. Part two involves something unheard of - changing your court. After enough moping around, you finally found two other pixies who feel like you. \cSAdvisor{} and \cMChange{} each have their reasons to want to switch to a different court, and it works out that you’d end up with Structure Magic. The Court of Structure is ideal. They are definitionally the most important Pixies. That’s messed up. But since you can’t fix the broken system, you might as well do what it takes to be on top.

\begin{itemz}[Goals (in roughly descending order of importance)]
	\item Find or replace the antidote for the Tree of Time. Administer the antidote sometime \textbf{after} \cMTree{} raises the alarm, and make sure everyone knows that \textbf{you} are the one rescuing the tree.
	\item Switch your magic with \cSAdvisor{} and \cMChange{} to become a Structure Pixie.
\end{itemz}

\begin{itemz}[Notes]
	\item Flora and Fauna pixies tend to understand ``goodbye'' better than most other pixies, since animals and plants don’t live forever the way pixies do. They still aren't very good at it.
	\item You can sit on the secret that you were the one that poisoned the Tree of Time pretty hard. As a player, you can expect one or more characters to interrogate you on the matter at some point. If there is 45 minutes left in game, and no one has mentioned the issue with the Tree to you, see a GM.
	\item If you choose to make a new antidote, you need to use 5 pipe cleaners to make a cage-like structure to contain at least 1 cotton ball. There must be enough room inside the cage for the cotton ball to move a bit; it can't just be squashed in there. You may choose to send the antidote out of game to be administered at any time after you have recovered the old one, or made a new one. Unless \textbf{you} make a big deal about it, it is probably that no one will notice.
\end{itemz}

\begin{contacts}
	\contact{\cFHead{}} The head of the Flora and Fauna Pixies tonight. You are pretty sure the only pixies who would be more upset with you for doing what agarics do - poisoning something - than \cFHead{} would be \cSHead{} and \cSAdvisor{}, the leaders of the entirety of Pixie Hollow.
	\contact{\cMTree{}} The Pixie Dust expert who seems to have finally noticed that something is wrong with the tree.
	\contact{\cSAdvisor{}} A Structure Pixie who wants to trade \cSAdvisor{\their} magic for Magic and Maker magic (you do not understand why anyone would give up Structure magic, but it works in your favor).
	\contact{\cMChange{}} A Magic and Maker pixie who wants to trade \cSAdvisor{\their} magic for Magic and Flora and Fauna magic (you do not understand why anyone would want Flora and Fauna magic, but it works in your favor).
\end{contacts}

\end{document}
