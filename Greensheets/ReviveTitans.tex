\documentclass[green]{PP}
\parindent=0pt
\begin{document}
\name{\gTitan{}}

This greensheet does not represent a ritual, and therefore cannot be interrupted like a normal ritual. Anyone with this greensheet can perform these steps, or cause these steps to be performed. We recommend you collaborate so as to avoid duplicating steps. This seemingly straightforward and minimally magical activity works because you have already made extensive preparations and spent many magical resources in service to this.

\begin{enumerate}
	\item Prepare a single vessel for holding a scrap of all four elements (small craft)
	\begin{enumerate}
		\item Fold an Origami Masu Box (Instructions available at the Court of Elements Table).
		\item OPTIONAL: If you are having fun, you can make a lid, or make 4 boxes instead of just 1, but this is entirely up to your enjoyment as players.
	\end{enumerate}
	\item Have the 4 element pixies that correspond to the titans (\cEHead{} = Air, \cELove{} = Water, \cETitan{} = Earth, \cEAirship{} = Fire) do a little magic effect that calls upon their element while you are holding the vessel you made in step 1. (Tell the pixies they are able to do this freely; no particular ability is required, nor the expenditure of resources if they are element pixies.)
	\begin{enumerate}
		\item A light pixie (\cESweet{}) may substitute for any \textbf{one} other element.
		\item Any 2 non-Element pixie may substitute for \textbf{one} of elements with the following constraints: 2 pixies must agree to do this, and they must expend 1 unit of pixie dust (or the equivalent amount of energy from the Solar Butterflies; ask a Flora and Fauna Pixie about it.) -- You may use this substitution multiple times (e.g. to substitute for both air and water) but no pixie may represent more than 1 element. In other words, the same pixie cannot substitute twice.
	\end{enumerate}
	\item While holding the fully charged box(es), someone must announce loudly the following: “We give back a spark, a drop, a pebble, a breeze, of what was given freely.”
	\item Send the box(es) out of game using “Sign 2.” The pikas have agreed to help transport the box(es) to where the Titan’s remnants rest.
	\item 30 minutes later, the ground will tremble, and the other characters in the game will become aware that something \textbf{big} is happening, and that it involves the Titans.
	\item \textbf{Four} pixies must agree to accept the power of one of each of the Titans before the end of the game. These pixies cannot be going on the away team, but do not need to have participated in the preparations above, and do not need to be element pixies, or match the element (e.g. an Air Pixie could take on the power of the fire titan and a Structure Pixie could take on the power of the water titan). It is up to the player of each character to decide whether their character is willing to do this; no one can be forced to do this, and no characters are written as predetermined to be willing or unwilling to do this.
	\begin{enumerate}
		\item If you cannot find a pixie for every Titans, the whole thing fails and none are revived.
		\item You must truthfully tell any pixie that asks that \textbf{your character doesn’t know }what will happen if they accept the power of the Titans, but it’s probably not good. Pixies are very small, and this is a lot of power.
		\item \textit{(OOC Note: while your character does not know the consequence of accepting the titan’s power, please warn \textbf{players} that accepting this power will require their character to \textbf{post game} go and live with the four Titans for the next millennium as they slowly wake up and figure out how to live again. A millennium is long enough that it gives even Pixies, who are immortal, pause. They would be somewhat isolated from other Pixies, who could only come visit occasionally, and could not attend things like celebrations of life, or request one of their own, during this time.) }
	\end{enumerate}
\end{enumerate}

\end{document}
