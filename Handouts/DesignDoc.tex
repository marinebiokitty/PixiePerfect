\documentclass[sheet]{PP}

\usepackage{graphicx}
\graphicspath{ {./images/} }
\usepackage{xcolor}
\usepackage{hyperref}
\usepackage{multicol}
\usepackage{ltablex}
\usepackage{tabularx}
\usepackage{indentfirst}
\renewcommand{\tabularxcolumn}[1]{m{#1}}
\setlength{\columnsep}{1cm}
%
%%% document-wide tweaks
%\interlinepenalty10000
%\setstretch{1}
%\def\mytype{Rules and Scenario}
%\lfoot{}\rfoot{}
%
\begin{document}
\centering{\title{Pixie Perfect Design Document}}

\section*{Game Style:}
This game is a “Secrets and Powers” (S&P) game. The author considers this to be a subtype of “lit-form” games. Characters are pre-written, plot is pre-seeded, and the space of plot resolution is pre-defined. 

Many characters will have secrets they wish to keep hidden. Other characters will be looking to uncover those secrets. It is true that secrets are often the most fun when they come out in game, but in a game like this, it is often a good idea to expend some effort toward keeping secrets until the last quarter of game or so, since the secret might come out another way, without you helping it along.

Many characters also have powers. These powers may be unique to that character alone, or shared by a group of characters. Often these powers will be necessary for addressing problems and conflict that arise in game, offering players the opportunity to kind of swoop in and be the hero of the hour. Don’t forget to share that spotlight around!

This game style also includes a variety of mechanics that determine how characters can interact with the world, and what the outcomes are for those interactions. This allows players to autonomously interact with the world without waiting for a GM to adjudicate.


\section*{Content Warnings:}
Death, loss, risk to (but not total destruction) of Neverland, rule breaking, predetermined fate, conflicting/mutually exclusive goals, new and established romances (the extent of their interest in this game is holding hands with their romantic interests).

\section*{Tone:}
This game has a cute, friendly, innocent premise, but the game itself is not. It is not a harsh, backstabbing, power-grab kind of game, but it \textbf{is} a game about pain, loss, guilt, difficult choices, the weight of responsibility, and being dissatisfied with your fate. Not all characters are unhappy with their lot at the beginning of the game, but many want to enact changes to improve things, but those changes will not all be universally well received.

Players are expected to treat the content of the game with respect and seriousness. Some amount of levity is appropriate, along with up to fairly unflappable optimism that things will work out. Some amount of short lived anger may be appropriate based on the situation. Players are asked \textbf{not} to introduce additional harshness or cruelty, any murder, torture, sex, non-consent of any flavor, or religion into this game. Players should \textbf{not} expect to \textbf{solve} problems through violence, even if their character might.


\section*{Accessibility and Safety:}
Players are more important than the game. While there are many, various accessibility mechanics that can be employed, it is important to find the most suitable tools for the particular game. For this particular game, in this particular style, the following accessibility mechanics will be employed:
\begin{itemz}[Before the Game]
	\item \textbf{The Casting Survey} - Please use the spaces provided in the casting survey to communicate with the event runner about any concerns you have regarding the content of the game, or content warnings that you may need to avoid. Not every character interacts with every theme, and players can be cast to characters that are more distant from potentially activating content as long as the game runner knows about it.
	\item \textbf{Take Care of Yourself }- Before the game, make sure you eat, sleep, hydrate, take any necessary medications, etc. Addressing your physical needs can improve your emotional and mental bandwidth for game.
\end{itemz}
\begin{itemz}[During the Game]
	\item \textbf{The Out of Game Symbol} - During the game you can make a fist with one hand and place it on your head. This indicates to other players and the GM(s) that you are out of character, and are interacting as a player instead. There are several common uses for this symbol. We have chosen to unify the various actions to one symbol to reduce the number of different ones players have to remember.
	\begin{itemize}
		\item Because a mechanic tells you to do this. Some mechanics may remove your character from the game space for some reason, and this is how you indicate that.
		\item To talk to a GM - GMs can help you if you are lost, stuck, not having fun, have a question, etc.
		\item To talk to a player - either to check in if they are doing okay (e.g. they are crying and you want to check if they are OOC upset, or just IC upset), or to ask for some support that you need (e.g. asking a player to talk more quietly.)
		\item To step away from the game for OOC reasons (e.g. an urgent phone call, visiting the restroom).
	\end{itemize}
\end{itemz}
\begin{itemz}[After the Game]
	\item \textbf{Afters} - Many players find that hanging out with players after the event to socialize and at food is often helpful in transitioning back to normal life. At Intercon this won’t be possible as many players have other events to rush off to, but we have dedicated a period of time *during* the 4 hour game block for cool-down.
	\item \textbf{Take Care of Yourself} - After the game, make sure you eat, sleep, hydrate, take any necessary medications, etc. Addressing your physical needs can improve your emotional and mental recovery from game.
	\item \textbf{Reach out to GMs or Players} - If you need help or support, reach out to a GM or to a player you are friends with.
\end{itemz}

Physical accessibility will be gladly addressed as needed, based on player needs and venue constraints.

\section*{Player Influence Over Story:}
Characters are pre-written for this game, with a defined back-story, and specific goals. Players are expected to read and understand the character sheet, and play to that character. Goals can grow and evolve during the game, based on new information, but players are requested to stay within the bounds of what has been defined. (e.g.: You cannot just decide that your character is secretly a new kind of pixie who can safely handle iron, or that they have a vendetta against another character just because your IRL friend is playing them.)

Players will generally have agency over their character’s story, bounded by certain mechanics. Crucially, there is no way to kill another character during the game (player consent to it cannot not overrule this. Pixies can’t be killed except by extended contact with lots of iron; of which there is no source). Smaller quantities or durations of contact with iron can cause temporary or permanent damage to pixies, but otherwise Pixies are immortal.

There are a few situations in which players will not have complete control over what happens to their character. The most common cases for this are:
\begin{itemize}
	\item \textbf{Another player uses an ability on your character} - Example: The current ruling Monarch can order a pixie to be on the Away Team with a binding mechanic. 
	\item \textbf{A mechanic imposes a limit or change on your character} - Example:  A mechanic represents your character doing something physically strenuous and exhausting by forbidding your player to run for a certain duration after completing the mechanic.
	\item \textbf{A secret piece of information is uncovered in the story that pertains to your character:} - Example: You open a mem-packet that describes something your character did during a gap in their memory.
\end{itemize}	

Such changes imposed upon your character can take your story in new and exciting directions you wouldn’t have otherwise thought to explore. Embrace the twists that the game introduces (through story pieces that other characters have, through mechanics, etc.) you may be surprised at how exciting and wonderful your story turns out.

\end{document}
