\documentclass[sheet]{PP}

\usepackage{graphicx}
\graphicspath{ {./images/} }
\usepackage{xcolor}
\usepackage{hyperref}
\usepackage{multicol}
\usepackage{ltablex}
\usepackage{tabularx}
\usepackage{indentfirst}
\renewcommand{\tabularxcolumn}[1]{m{#1}}
\setlength{\columnsep}{1cm}
%
%%% document-wide tweaks
%\interlinepenalty10000
%\setstretch{1}
%\def\mytype{Rules and Scenario}
%\lfoot{}\rfoot{}
%
\begin{document}
\centering{\title{Pixie Perfect Rules Document}}

\section*{Real Life Safety:}
Both Players and GMs are more important than the game. We will use the following safety tools in this game. We will go over how each of these tools will be used in this game as part of the briefing portion of the event: 
\begin{itemize}
  \item Out of Character Symbol (fist on head by default; will serve similar purpose to ``look down,'' and ``Cut''/``Brake'' as well as being generic out of character indicator.)
  \item Open Door Policy (you can always leave, temporarily or permenantly.)
\end{itemize}

\section*{Interrupting someone:}
Actions cannot generally be interrupted. A character does a thing, and it happens. The only exception is ``rituals'' (see below). Rituals are a single activity that requires multiple steps. A character can interrupt a ritual by moving close enough that you could reach out and touch the other player. Point at them and say ``I stop you''. No touching required. \textit{(Fun fact: If you know what a ZoC (zone of control) is, this is HALF that distance.)}

\section*{Magic:}
Big magic in this game is represented as greensheets. Assume your character can do small, harmless/ minor cosmetic effects that you judge would be in your purvew. For example, a fire pixie can light a candle, a structure pixie can emit a calming aura that may or may not help (at the discretion of the target), a Maker Pixie can repair a dent in the wall. If you want to do something with mechanical consequences to characters or the game world, see a GM. Some magic will require consuming a unit of \textbf{Pixie Dust}, or another source of magic (at a 2:1 ratio). \textbf{Pixie Dust} is a specific, consumable item that Magic and Maker pixies are the most likely to have on hand.

\subsection*{Rituals:}
Most big magic take place as rituals. These are usually described in greensheets and involve several steps to complete. You must always answer truthfully if someone asks if you are performing a ritual (it may not be obvious to players, but it is always obvious to characters). Rituals are always interruptible (see above). Some rituals will require you to start over if you are interrupted. Others are just paused. Check your ritual for specifics.

\section*{Combat:}
Violence is quite uncommon among pixies, but sometimes you just can't talk things out, even with the help of a friend. You may need to restrain someone or something, take something away from someone, or in the most extreme cases, knock something unconscious (e.g. a rampaging squirrel).

Combat is a straightforward matter of playing rock-paper-scissors (Please don't game the system; play it straight). Use this mechanic if you encounter a situation that cannot be resolved through IC discussion in which the involved characters agree on a course of action. You may attempt to force your course of action by invoking combat. We will go over the exact rules for rock-paper-scissors during briefing because it is much easier to show than write out.

\section*{Stealth:} There is no stealth mechanic for combat in this game.

\section*{Health State and Death:} You cannot instigate combat to kill another character. \textbf{The most harm you can do is to knock someone out.} Pixies will find such violence excessive in all but the most dire of circumstances, and others may react poorly to violence.

A knocked out character will wake up in 60 seconds. A restrained character will escape their restraints in 2 minutes, unless someone's full attention is on keeping them restrained.

\end{document}
