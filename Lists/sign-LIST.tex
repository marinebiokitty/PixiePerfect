%%%%
%%
%% This file sets up the Sign and Label datatypes and creates Sign and
%% Label macros.
%%
%% Signs generally represent interesting parts of game area, usually
%% as things posted on walls.  Labels represent other things, often on
%% or inside envelopes, that are part of complex mechanics.
%%
%% The default value for \MYloc will inherit location from the Place
%% or Sign most immediately up the ownership tree.  Override this by
%% setting \MYloc to anything (even blank).
%%
%% Sign is for full-sized signs that would cover most of a large
%% manila envelope; SignMedium is for signs sized to half-sized manila
%% envelopes; SignSmall is for signs sized for small manila envelopes
%% (the same size as item cards).  Label, LabelMedium, and LabelSmall
%% are analagous, but they don't have a \takedownby note at the
%% bottom.  You can always use a sign or label without an envelope or
%% with a differently-sized envelope.  Choose which based on
%% visibility and content.
%%
%% SignTiny is for signs you want to be hard to find; it is small and
%% does not have a \takedownby note.  SignDot is for a very small
%% "dot" which only has a title.
%%
%% SignStrip produces a strip of paper (without a \takedownby note)
%% with labels on the outside that show on both sides if you fold it
%% in half.  These are a convenient alternative to sub-envelopes. They
%% can also be used for "s-packets" taped to walls (see
%% Extras/README-s-packets).
%%
%% LabelCover produces a label similar to the cover to a research
%% notebook.  LabelPage, likewise, produces a page.
%%
%% EOG is for full-sized end-of-game signs.
%%
%%%%%

\DECLARESUBTYPE{Sign}{Element}
\PRESETS{Sign}{
  \FD\MYloc	{\mylocation} %% real-space location
  \FD\MYtext	{} %% text of sign
  }
\POSTSETS{Sign}{
  \edef\mylocation{\MYloc}
  \protected@edef\@ownerstring{%
    \MYname%
    \ifx\mylocation\empty\else\ (\mylocation)\fi%
    }
  }
\def\mylocation{}

\def\loc#1{\rs\MYloc{#1}}

\DECLARESUBTYPE{SignMedium}{Sign}
\DECLARESUBTYPE{SignSmall}{Sign}
\DECLARESUBTYPE{SignTiny}{Sign}
\DECLARESUBTYPE{SignDot}{Sign}
\PRESETS{SignDot}{\s\MYtext{}}

\DECLARESUBTYPE{Label}{Sign}
\PRESETS{Label}{\s\MYloc{}}
\DECLARESUBTYPE{LabelMedium}{Label}
\DECLARESUBTYPE{LabelSmall}{Label}

\DECLARESUBTYPE{SignStrip}{Sign}
\DECLARESUBTYPE{LabelCover}{Label}
\DECLARESUBTYPE{LabelPage}{Label}

\DECLARESUBTYPE{EOG}{Sign}
\PRESETS{EOG}{%
  \s\MYname	{End Of Game}
  \s\MYtext	{{\bf\Huge You may not pass through here.}}
  }


%%%%%
%% \signbig[<location>]{<name>}{<text>}
%% \eog[<location>]
%%
%% \signmdeium[<location>]{<name>}{<text>}
%% \signsmall[<location>]{<name>}{<text>}
%% \signtiny[<location>]{<name>}{<text>}
%% \signdot[<location>]{<name>}
%%
%% \labelbig{<name>}{<text>}
%% \labelmedium{<name>}{<text>}
%% \labelsmall{<name>}{<text>}
%%
%% \signstrip[<location>]{<name>}{<text>}
%% \labelcover{<name>}{<text>}
%% \labelpage{<name>}{<text>}
\newinstance{Sign}{\signbig[3][\mylocation]}{
  \s\MYloc{#1}\s\MYname{#2}\s\MYtext{#3}}
\newinstance{EOG}{\eog[1][\mylocation]}{\s\MYloc{#1}}

\newinstance{SignMedium}{\SignMedium[3][\mylocation]}{
  \s\MYloc{#1}\s\MYname{#2}\s\MYtext{#3}}
\newinstance{SignSmall}{\signsmall[3][\mylocation]}{
  \s\MYloc{#1}\s\MYname{#2}\s\MYtext{#3}}
\newinstance{SignTiny}{\signtiny[3][\mylocation]}{
  \s\MYloc{#1}\s\MYname{#2}\s\MYtext{#3}}
\newinstance{SignDot}{\signdot[2][\mylocation]}{
  \s\MYloc{#1}\s\MYname{#2}}

\newinstance{Label}{\labelbig[2]}{
  \s\MYname{#1}\s\MYtext{#2}}
\newinstance{LabelMedium}{\labelmedium[2]}{
  \s\MYname{#1}\s\MYtext{#2}}
\newinstance{LabelSmall}{\labelsmall[2]}{
  \s\MYname{#1}\s\MYtext{#2}}

\newinstance{SignStrip}{\signstrip[3][\mylocation]}{
  \s\MYloc{#1}\s\MYname{#2}\s\MYtext{#3}}
\newinstance{LabelCover}{\labelcover[2]}{
  \s\MYname{#1}\s\MYtext{#2}}
\newinstance{LabelPage}{\labelpage[2]}{
  \s\MYname{#1}\s\MYtext{#2}}


%%%%%
%% \sEOG{}
%% use \sEOg[\loc{<location>}]{} for EOG sign at a specific place
\NEW{EOG}{\sEOG}{
  }


%%%%%%%%%%%%%%%%%%%%%%%%%%%%%%%%%%%%%%%%%%%%%%%%%%%%%%%%%%%%%%%%%%

\NEW{Sign}{\sTest}{
  \s\MYname	{A Room}
  \s\MYloc	{10-250}
  \s\MYtext	{A lecture hall with large, sliding blackboards.}
}


%%%%%%%%%%%%%%%%%%%%%%%%%%%%%%%%%%%%%%%%%%%%%%%%%%%%%%%%%%%%%%%%%%

\NEW{Sign}{\sSignOne}{
  \s\MYname	{Sign 1}
  \s\MYloc	{Mortal World}
  \s\MYtext	{{\large \textbf{Do not interact with this sign unless a greensheet tells you to do so.}}
	
	Your default timer is 5 minutes, unless you have been told where to find what you are looking for, in which case your timer is 2 minutes. You have to hang out on this side of the portal until the timer is done. You may talk to other characters, and even help them with tasks, but if you choose to engage with anyone, you must tell them what you are doing. If you choose not to engage with anyone, you can just tell them you are ``not here''.}
	\s\MYitems {\multi{6}{\iIron{}}}
}

\NEW{Sign}{\sSignTwo}{
  \s\MYname	{Sign 2}
  \s\MYloc	{Neverland}
  \s\MYtext	{\large{\textbf{Do not interact with this sign unless a greensheet tells you to do so.}}
	
	You may drop the required item(s) off here once you have completed all required steps.}
}

\NEW{Sign}{\sSignThree}{
  \s\MYname	{Sign 3}
  \s\MYloc	{Near Mortal World}
  \s\MYtext	{If flowers are present near this sign (e.g. placed on a chair next to it), they will attract Solar Butterflies from all over the Mortal World. These are represented by GOLD plastic butterflies. Up to 6 butterflies may be attracted to each flower at a time. All of the Flora and Fauna pixies should know what to do with them but \cFButterfly{} is in charge. Please don't touch them without directions from one of them.
	
	If you wish to draw magical energy from a \textbf{Solar Butterfly}, you must talk to \cFHead{} or \cFButterfly{}. They are the \textbf{only ones }who can authorize the use of this magical energy. The energy is available for use, even if the phys reps have been picked up.
	
	If you are directed to do so, you may place SILVER plastic butterflies here; the GM will pick them up and update the count below.
	
	\large{Number of Lunar Moths sent to the Mortal World:} }
}

\NEW{Sign}{\sSignFour}{
  \s\MYname	{Sign 4}
  \s\MYloc	{Neverland}
  \s\MYtext	{If flowers are present near this sign (e.g. placed on a chair next to it), they will attract Lunar Moths from all over Neverland. These are represented by SILVER plastic butterflies. Up to 6 moths may be attracted to each flower at a time. All of the Flora and Fauna pixies should know what to do with them but \cFButterfly{} is in charge. Please don't touch them without directions from one of them.
	
	The Lunar Moths have dispersed their magical energy across Neverland; they cannot be used as a source of magical energy. If you wish to draw magical energy from a \textbf{Solar Butterfly}, you must talk to \cFHead{} or \cFButterfly{}. They are the \textbf{only ones} who can authorize the use of this magical energy. The energy is available for use, even if the phys reps have been picked up.
	
	If you are directed to do so, you may place GOLD plastic butterflies here; the GM will pick them up and update the count below.
	
	\large{Number of Solar Butterflies Returned to Neverland:} }
}

\NEW{Sign}{\sSignFive}{
  \s\MYname	{Sign 5}
  \s\MYloc	{The Portal}
  \s\MYtext	{	\large{The Portal between Neverland and the Mortal World will be:}
	\begin{enumerate}
		\item Closed but translucent for the first 15 minutes of game. Pixies can see silhouettes through it, but cannot identify individual pixies.
		\item Open between T+15 from game start to T-5 from game end. Pixies can pass freely through the portal back and forth, carrying whatever they would like.
		\item Closed but translucent for the last 5 minutes of game. The new Away Team, consisting of exactly 4 pixies - exactly one from each court - must be on the mortal side of the portal before it closes, and must stay there to distribute the Seasons. The portal will not open again for 2 years. If the away team is not composed of one pixie from each court, the seasons will not function properly, and the Mortal world will suffer. In turn, the Lunar Moths will not be able to collect as much magic to return to Neverland, and Neverland will suffer in the future.
	\end{enumerate}
	}
}

\newcommand{\resourcestation}[4]{
	\NEW{Sign}{#1}{
		\s\MYname{Resource Station #2}
		\s\MYloc{}
		\s\MYtext{
			At any point, \textbf{any #3 pixie} can approach this station.  
			
			If they do so and the timer is not running or finished running, they may restart the timer and take \textbf{#4}.  
			
			If the timer is still running, they must come back later.
		}
	}
}

\resourcestation{\sResourceStructure}{of the Court of Structure}{Structure}{two units of paper, both of which must be of the same kind: origami, construction, tissue, or round white circles}
\resourcestation{\sResourceFF}{of the Court of Flora and Fauna}{Flora and Fauna}{either one stamp and one stamp pad, or five leaf stickers}
\resourcestation{\sResourceMakers}{for Makers}{Maker}{either three units of pipe cleaners, or three cotton balls}

\NEW{Sign}{\sResourceMagic}{
	\s\MYname{Resource Station for Magic}
	\s\MYloc{}
	\s\MYtext{
		At any point, \textbf{any Magic pixie (of the Court of Makers and Magic)} can take any quantity of pixie dust from the envelope below.
	}
	\s\MYitems{\multi{4}{\iPixieDust{}}}
}
\NEW{Sign}{\sResourceElements}{
	\s\MYname{Resource Station for the Court of Elements}
	\s\MYloc{}
	\s\MYtext{
		At any point, \textbf{any Elements pixie} can take pens, glue, scissors, or instruction sheets from this station.
	}
}